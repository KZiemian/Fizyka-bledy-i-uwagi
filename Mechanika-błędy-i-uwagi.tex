% Autor: Kamil Ziemian

% --------------------------------------------------------------------
% Podstawowe ustawienia i pakiety
% --------------------------------------------------------------------
\RequirePackage[l2tabu, orthodox]{nag} % Wykrywa przestarzałe i niewłaściwe
% sposoby używania LaTeXa. Więcej jest w l2tabu English version.
\documentclass[a4paper,11pt]{article}
% {rozmiar papieru, rozmiar fontu}[klasa dokumentu]
\usepackage[MeX]{polski} % Polonizacja LaTeXa, bez niej będzie pracował
% w języku angielskim.
\usepackage[utf8]{inputenc} % Włączenie kodowania UTF-8, co daje dostęp
% do polskich znaków.
\usepackage{lmodern} % Wprowadza fonty Latin Modern.
\usepackage[T1]{fontenc} % Potrzebne do używania fontów Latin Modern.



% ----------------------------
% Podstawowe pakiety (niezwiązane z ustawieniami języka)
% ----------------------------
\usepackage{microtype} % Twierdzi, że poprawi rozmiar odstępów w tekście.
\usepackage{graphicx} % Wprowadza bardzo potrzebne komendy do wstawiania
% grafiki.
\usepackage{verbatim} % Poprawia otoczenie VERBATIME.
\usepackage{textcomp} % Dodaje takie symbole jak stopnie Celsiusa,
% wprowadzane bezpośrednio w tekście.
\usepackage{vmargin} % Pozwala na prostą kontrolę rozmiaru marginesów,
% za pomocą komend poniżej. Rozmiar odstępów jest mierzony w calach.
% ----------------------------
% MARGINS
% ----------------------------
\setmarginsrb
{ 0.7in} % left margin
{ 0.6in} % top margin
{ 0.7in} % right margin
{ 0.8in} % bottom margin
{  20pt} % head height
{0.25in} % head sep
{   9pt} % foot height
{ 0.3in} % foot sep



% ------------------------------
% Często używane pakiety
% ------------------------------
\usepackage{csquotes} % Pozwala w prosty sposób wstawiać cytaty do tekstu.
\usepackage{xcolor} % Pozwala używać kolorowych czcionek (zapewne dużo
% więcej, ale ja nie potrafię nic o tym powiedzieć).



% ------------------------------
% ------------------------------
% Pakiety do tekstów z nauk przyrodniczych
% ------------------------------
\let\lll\undefined % Amsmath gryzie się z językiem pakietami do języka
% polskiego, bo oba definiują komendę \lll. Aby rozwiązać ten problem
% oddefiniowuję tę komendę, ale może tym samym pozbywam się dużego Ł.
\usepackage[intlimits]{amsmath} % Podstawowe wsparcie od American
% Mathematical Society (w skrócie AMS)
\usepackage{amsfonts, amssymb, amscd, amsthm} % Dalsze wsparcie od AMS
\usepackage{siunitx} % Dla postrzego pisania jednostek fizycznych
\usepackage{upgreek} % Ładniejsze greckie litery
% Przykładowa składnia: pi = \uppi
\usepackage{slashed} % Pozwala w prosty sposób pisać slash Feynmana.
\usepackage{calrsfs} % Zmienia czcionkę kaligraficzną w \mathcal
% na ładniejszą. Może w innych miejscach robi to samo, ale o tym nic
% nie wiem.



% ##########
% Tworzenie otoczeń "Twierdzenie", "Definicja", "Lemat", etc.
\newtheorem{twr}{Twierdzenie} % Komenda wprowadzająca otoczenie
% ,,twr'' do pisania twierdzeń matematycznych
\newtheorem{defin}{Definicja} % Analogicznie jak powyżej
\newtheorem{wni}{Wniosek}



% --------------------------------------------------------------------
% Dodatkowe ustawienia dla języka polskiego
% --------------------------------------------------------------------
\renewcommand{\thesection}{\arabic{section}.}
% Kropki po numerach rozdziału (polski zwyczaj topograficzny)
\renewcommand{\thesubsection}{\thesection\arabic{subsection}}
% Brak kropki po numerach podrozdziału



% ----------------------------
% Ustawienia różnych parametrów tekstu
% ----------------------------
\renewcommand{\arraystretch}{1.2} % Ustawienie szerokości odstępów między
% wierszami w tabelach.





% --------------------------------------------------------------------
% Różne komendy zdefiniowane dla ułatwienia pracy z LaTeXem
% --------------------------------------------------------------------

% ##########
% Definicje odstępów w tekście
\newcommand{\spaceOne}{3em}
\newcommand{\spaceTwo}{2em}
\newcommand{\spaceThree}{1em}
\newcommand{\spaceFour}{0.5em}



% ##########
% Skróty do często używanych komend
\newcommand{\ld}{\ldots}
\newcommand{\tbs}{\textbackslash}  % Backslash w tekście

% ##########
% Podstawowa komendy do edycja tekstu
\newcommand{\tb}{\textbf}
\newcommand{\noi}{\noindent}

% ##########
% Skróty do często używanych komend
\newcommand{\ld}{\ldots}
\newcommand{\tbs}{\textbackslash} % Backslash w tekście

% ##########
% Podstawowa edycja tekstu (może jest na to lepsze nazwa?)
\newcommand{\tb}{\textbf}
\newcommand{\noi}{\noindent}
\newcommand{\start}{\noi \tb{--} {}}

% Numeracja stron i rozdziałów
\newcommand{\Str}[1]{\tb{Str. #1.}}
\newcommand{\StrWg}[2]{\tb{Str. #1, wiersz #2.}}
\newcommand{\StrWd}[2]{\tb{Str. #1, wiersz #2 (od dołu).}}

% Do poprawiania błędów
\newcommand{\Dow}{\tb{Dowód}} % To niezbyt pasuje do opisu POPRAW
\newcommand{\Jest}{\tb{Jest: }}
\newcommand{\Pow}{\tb{Powinno być: }}

% Nagłówki
\newcommand{\Center}[1]{\begin{center} #1 \end{center}}
\newcommand{\CenterTB}[1]{\Center{\tb{#1}}}

\newcommand{\Main}[1]{ \begin{center} {\LARGE \tb{#1} } \end{center} }
\newcommand{\Field}[1]{ \begin{center} {\Large \tb{#1} } \end{center} }
\newcommand{\Work}[1]{ \begin{center} {\large \tb{#1}} \end{center} }





% --------------------------------------
% Komendy do trybu matematycznego
% --------------------------------------


% Ładniejszy zbiór pusty
\let\oldemptyset\emptyset
\let\emptyset\varnothing


% Podstawowe oznaczenia matematyczne
\newcommand{\fr}{\frac}
\newcommand{\tfr}{\tfrac}
\newcommand{\tr}{\textrm}

% Ułamki
\newcommand{\onehalf}{\fr{ 1 }{ 2 }}
\newcommand{\ovOne}[1]{\fr{ 1 }{ #1 }}
\newcommand{\ovTwo}[1]{\fr{ #1 }{ 2 }}

% Oznaczenia ,,nad i pod''. Wymyśl lepszą nazwę
\newcommand{\til}{\tilde}
\newcommand{\ul}{\underline}
\newcommand{\ol}{\overline}
\newcommand{\wh}{\widehat}
\newcommand{\wt}{\widetilde}

% Trzcionki matematyczne
\newcommand{\mr}{\mathrm}
\newcommand{\mb}{\mathbb}
\newcommand{\mc}{\mathcal}
\newcommand{\mf}{\mathfrak}
\newcommand{\mbf}{\mathbf}

% Strzałki
\newcommand{\ra}{\rightarrow}
\newcommand{\Ra}{\Rightarrow}
\newcommand{\lra}{\longrightarrow}
\newcommand{\xra}{\xrightarrow}
\newcommand{\ifonlyif}{\Longleftrightarrow}

% Częste zwroty
\newcommand{\wtw}{wtedy i~tylko wtedy}


% Wartość bezwzględna i normy
\providecommand{\absj}[1]{\lvert #1 \rvert}
\providecommand{\absd}[1]{\left| #1 \right|}
\providecommand{\absd}[1]{\left| \, #1 \, \right|}
% Zupełnie nie wiem, czemu to jest \providecommand
\newcommand{\norm}[1]{\left|\left| #1 \right|\right|}


% ##########
% Litery greckie
\newcommand{\al}{\alpha}
\newcommand{\be}{\beta}
\newcommand{\ga}{\gamma}
\newcommand{\Ga}{\Gamma}
\newcommand{\del}{\delta}
\newcommand{\Del}{\Delta}
\newcommand{\la}{\uplambda}
\newcommand{\eps}{\epsilon}
\newcommand{\veps}{\varepsilon}
\newcommand{\vp}{\varphi}
\newcommand{\om}{\omega}
\newcommand{\Om}{\Omega}
\newcommand{\si}{\sigma}
\newcommand{\Si}{\Sigma}
\newcommand{\tet}{\theta}


% ##########
% Standardowe używane w oznaczeniach litery o specjalnej czcionce
% \mb = \mathbb
\newcommand{\Cc}{\mc{C}}
\newcommand{\Dc}{\mc{D}}
\newcommand{\Hc}{\mc{H}}
\newcommand{\lc}{\mc{l}}
\newcommand{\Lc}{\mc{L}}
\newcommand{\N}{\mb{N}}
\newcommand{\R}{\mb{R}}
\newcommand{\Rn}{\R^{ n }}
\newcommand{\Rc}{\mc{R}}

\newcommand{\Rp}{\R_{ + }}


% Rzadziej używane oznaczenia literowe
% \mc = \mathcal
\newcommand{\B}{\mc{B}}
\newcommand{\Fc}{\mc{F}}
\newcommand{\M}{\mc{M}}
\newcommand{\Oc}{\mc{O}}
\newcommand{\T}{\mc{T}}
\newcommand{\Pc}{\mc{P}}

\newcommand{\Rt}{\R^{ 3 }}



% ##########
% Teoria mnogości
\newcommand{\set}[1]{\{ #1 \}}
\newcommand{\es}{\emptyset}
\newcommand{\subs}{\subset}
\newcommand{\setm}{\setminus}
\newcommand{\ti}{\times}


% ##########
% Algebra
\newcommand{\Real}{\mf{Re}}
\newcommand{\Imag}{\mf{Im}}
\newcommand{\ot}{\otimes}
\newcommand{\Tr}{\mr{Tr}}



% ##########
% Analiza matematyczna

% Granice
\newcommand{\Lim}{\lim\limits}
\newcommand{\Liminf}{\ul{\lim}}
\newcommand{\Limsup}{\ol{\lim}}

\newcommand{\nToInfty}{n \ra +\infty}

% Sumy
\newcommand{\Sum}{\sum\limits}

% Różniczkowanie i pochodne
\newcommand{\Cinfty}{\Cc^{ \infty }}

\newcommand{\de}{\mr{d}}
\newcommand{\dd}[3]{\fr{ \de^{ #1 } { #2 } }{ \de { #3 }^{ #1 } }}

\newcommand{\pr}{\partial}
\newcommand{\pd}[3]{\fr{ \pr^{ #1 } { #2 } }{ \pr { #3 }^{ #1 } }}

% Całki i całkowanie
\newcommand{\dx}{\de x}
\newcommand{\dk}{\, d}  % Różniczka Kończąca całkę

\newcommand{\sqrtTwo}{\sqrt{ 2 }}
\newcommand{\ovSqTwo}{\fr{ 1 }{ \sqrtTwo }}
\newcommand{\sqTwoPi}{\sqrt{ 2 \pi }}
\newcommand{\ovSqTwoPi}{\fr{ 1 }{ \sqrt{ 2 \pi } }}

\newcommand{\Int}{\int\limits}
\newcommand{\IntSet}[3]{\Int_{ #1 } #2 \, d#3} % Całka po zbiorze.
\newcommand{\IntL}[3]{ \Int_{ { #1 } }^{ { #2 } } \de#3 \, }
\newcommand{\IntCa}[2]{\Int #1 \, \de#2} % Juz z calkowaniem
\newcommand{\IntFi}[2]{\Int \de#1 \, #2} % Bardziej Fizyczna notacja;)
\newcommand{\IntWie}[3]{\Int_{ #1 } \de^{ #2 }#3 \,} % Wielowymiarowa
\newcommand{\IntTro}[1]{ \Int \de^{ 3 }#1 \,} % Trójwymiarowa
\newcommand{\IntDo}[5]{ \Int_{ #1 } \de^{ #2 }#3\, \de^{ #4 } #5 \; }



% ##########
% Analiza funkcjonalna
\newcommand{\da}{\dagger}

% Brakety (w analizie funkcjonalnej też się je często stosuje)
\newcommand{\lket}{\langle}
\newcommand{\rket}{\rangle}

% Przestrzenie L^{ p }
\newcommand{\Lj}{L^{ 1 }}
\newcommand{\Ld}{L^{ 2 }}
\newcommand{\Lp}{L^{ p }}
\newcommand{\LIj}{\Lc^{ 1 }}
\newcommand{\LIp}{\Lc^{ p }}
\newcommand{\LdJ}{\Ld( \R, \de \mu )}
\newcommand{\LdT}{\Ld( \R^{ 3 }, \de \mu )}
\newcommand{\Ldlim}{L^{ 2 }-\Lim}




% ########################################





% ####################
% Analiza matematyczna


% Całki i całkowanie
\newcommand{\Int}{\int\limits}
\newcommand{\IntL}[3]{ \Int_{ { #1 } }^{ { #2 } } \de#3 \; }
\newcommand{\IntCaJ}[2]{ \Int #1 \, \de#2 } % Juz z calkowaniem
% Całka Jeden
\newcommand{\IntCaD}[2] { \Int #1 \, d#2 } % Całka dwa
\newcommand{\IntFi}[2]{ \Int \de#1 \, #2 } % Bardziej Fizyczna notacja;)
\newcommand{\IntWie}[3]{ \Int_{ #1 } \de^{ #2 }#3 \; } % Wielowymiarowa
\newcommand{\IntDo}[5]{ \Int_{ #1 } \de^{ #2 }#3\, \de^{ #4 } #5 \; }

% ####################
% Całki w określonych granicach
\newcommand{\IntA}[1]{\Int_{ -\infty }^{ +\infty } \de #1 \;}
\newcommand{\IntB}[1]{\int_{ \R } \de #1 \;}
\newcommand{\IntC}[2]{\int_{ \R } #1 \, \de #2}
\newcommand{\IntD}[1]{\Int_{ 0 }^{ +\infty } \de #1 \;}




% ####################
% Edycja tekstu
\newcommand{\Jest}{\tb{Jest: }}
\newcommand{\Pow}{\tb{Powinno być: }}
\newcommand{\red}[1]{{\color{red} #1}}
\newcommand{\Prze}{{\color{red} Przemyśl.}}
\newcommand{\Pop}{{\color{red} Popraw.}}
\newcommand{\Prob}{{\color{red} Problem.}}
\newcommand{\Dok}{{\color{red} Dokończ.}}
\newcommand{\Pyt}{{\color{red} Pytanie.}}
\newcommand{\Main}[1]{ \begin{center} {\LARGE \tb{#1} } \end{center} }
\newcommand{\Field}[1]{ \begin{center} {\Large \tb{#1} } \end{center} }
\newcommand{\Work}[1]{ \begin{center} {\large \tb{#1}} \end{center} }





% --------------------------------------------------------------------
% Komendy zdefiniowane konkretnie dla tego pliku:
% Mechanika-błędy-i-uwagi.tex
% --------------------------------------------------------------------

% ##########
% Oznaczenia dla książki Skalmierskiego

\newcommand{\eb}{\bd{e}} % e boldsymbol
\newcommand{\ebd}{\dot{ \eb }} % e boldsymbol dot

\newcommand{\vb}{\bd{v}} % v boldsymbol

% Wersory
\newcommand{\ib}{\bd{i}}
\newcommand{\jb}{\bd{j}}
\newcommand{\kb}{\bd{k}}

\newcommand{\dotx}{\dot{ x }}



% Koniec komend
% ############################





% ----------------------------
% Pakiet "hyperref"
% Polecano by umieszczać go na końcu preambuły.
% ----------------------------
\usepackage{hyperref} % Pozwala tworzyć hiperlinki i zamienia odwołania
% do bibliografii na hiperlinki.\RequirePackage[l2tabu, orthodox]{nag}







% Oznaczenia ,,nad i pod''. Wymyśl lepszą nazwę
\newcommand{\til}{\tilde}
\newcommand{\ul}{\underline}
\newcommand{\ol}{\overline}
\newcommand{\wh}{\widehat}
\newcommand{\wt}{\widetilde}

% Trzcionki matematyczne
\newcommand{\mr}{\mathrm}
\newcommand{\mb}{\mathbb}
\newcommand{\mc}{\mathcal}
\newcommand{\mf}{\mathfrak}
\newcommand{\mbf}{\mathbf}

% Strzałki
\newcommand{\ra}{\rightarrow}
\newcommand{\Ra}{\Rightarrow}
\newcommand{\lra}{\longrightarrow}
\newcommand{\xra}{\xrightarrow}

\newcommand{\wtw}{wtedy i~tylko wtedy}


% ####################
% Litery greckie
\newcommand{\al}{\alpha}
\newcommand{\be}{\beta}
\newcommand{\ga}{\gamma}
\newcommand{\del}{\delta}
\newcommand{\Del}{\Delta}
\newcommand{\la}{\uplambda}
\newcommand{\eps}{\epsilon}
\newcommand{\veps}{\varepsilon}
\newcommand{\vp}{\varphi}
\newcommand{\om}{\omega}
\newcommand{\Om}{\Omega}
\newcommand{\si}{\sigma}
\newcommand{\Si}{\Sigma}
\newcommand{\tet}{\theta}

% Standardowe oznaczenia literowe
\newcommand{\N}{\mb{N}}
\newcommand{\R}{\mb{R}}
\newcommand{\C}{\mb{C}}
\newcommand{\D}{\mc{D}}
\newcommand{\Hc}{\mc{H}}
\newcommand{\Lc}{\mc{L}}
\newcommand{\Rn}{\R^{ n }}
\newcommand{\Rc}{\mc{R}}
\newcommand{\Cc}{\mc{C}}
\newcommand{\lc}{\mc{l}}

% Mniej używane oznaczenia literowe
\newcommand{\B}{\mc{B}}
\newcommand{\Oc}{\mc{O}}
\newcommand{\Rp}{\R_{ + }}

% ####################
% Standardowe matematyczne symbole literowe
% Mathbb
% \newcommand{\C}{\mb{C}}
% \newcommand{\N}{\mb{N}}
% \newcommand{\R}{\mb{R}}
% \newcommand{\Rn}{\R^{ n }}
% \newcommand{\Rp}{\R_{ + }}

% % Mathcal
% \newcommand{\B}{\mc{B}}
% \newcommand{\D}{\mc{D}}
% \newcommand{\Fc}{\mc{F}}
% \newcommand{\M}{\mc{M}}
% \newcommand{\Rc}{\mc{R}}
% \newcommand{\Cc}{\mc{C}}
% \newcommand{\Oc}{\mc{O}}
% \newcommand{\T}{\mc{T}}


% ####################
% Teoria mnogości
\newcommand{\set}[1]{\{ #1 \}}
\newcommand{\es}{\emptyset}
\newcommand{\sset}{\subset}
\newcommand{\setm}{\setminus}
\newcommand{\ti}{\times}


% ####################
% Algebra
\newcommand{\Real}{\mf{Re}}
\newcommand{\Imag}{\mf{Im}}
\newcommand{\ot}{\otimes}
\newcommand{\Tr}{\mr{Tr}}


% ####################
% Analiza matematyczna

% Granice
\newcommand{\Lim}{\lim\limits}
\newcommand{\Liminf}{\ul{\lim}}
\newcommand{\Limsup}{\ol{\lim}}

% Sumy
\newcommand{\Sum}{\sum\limits}

% Różniczkowanie i pochodne
\newcommand{\dk}{\, d} % Różniczka Kończąca całkę
\newcommand{\pr}{\partial}
\newcommand{\de}{\mr{d}}
\newcommand{\dd}[3]{\fr{ \de^{ #1 } { #2 } }{ \de { #3 }^{ #1 } }}
\newcommand{\pd}[3]{\fr{ \pr^{ #1 } { #2 } }{ \pr { #3 }^{ #1 } }}

% Powszechnie używane symbole
\newcommand{\Cinfty}{\Cc^{ \infty }}

% Całki i całkowanie
\newcommand{\dx}{\de x}
\newcommand{\Int}{\int\limits}
\newcommand{\IntSet}[3]{\Int_{ #1 } #2 \, d#3} % Całka po zbiorze.
\newcommand{\IntL}[3]{ \Int_{ { #1 } }^{ { #2 } } \de#3 \; }
\newcommand{\IntCaJ}[2]{ \Int #1 \, \de#2 } % Juz z calkowaniem
% Całka Jeden
\newcommand{\IntCaD}[2] { \Int #1 \, d#2 } % Całka dwa
\newcommand{\IntFi}[2]{ \Int \de#1 \, #2 } % Bardziej Fizyczna notacja;)
\newcommand{\IntWie}[3]{ \Int_{ #1 } \de^{ #2 }#3 \; } % Wielowymiarowa
\newcommand{\IntDo}[5]{ \Int_{ #1 } \de^{ #2 }#3\, \de^{ #4 } #5 \; }

% ####################
% Całki w określonych granicach
\newcommand{\IntA}[1]{\Int_{ -\infty }^{ +\infty } \de #1 \;}
\newcommand{\IntB}[1]{\int_{ \R } \de #1 \;}
\newcommand{\IntC}[2]{\int_{ \R } #1 \, \de #2}
\newcommand{\IntD}[1]{\Int_{ 0 }^{ +\infty } \de #1 \;}


% ####################
% Analiza funkcjonalna
\newcommand{\da}{\dagger}

% Brakety
% W analizie funkcjonalnej też często się je stosuje
\newcommand{\lket}{\langle}
\newcommand{\rket}{\rangle}

% Przestrzenie L^{ p }
\newcommand{\Lj}{L^{ 1 }}
\newcommand{\Ld}{L^{ 2 }}
\newcommand{\Lp}{L^{ p }}
\newcommand{\LIj}{\Lc^{ 1 }}
\newcommand{\LIp}{\Lc^{ p }}
\newcommand{\LdJ}{\Ld( \R, \de \mu )}
\newcommand{\LdT}{\Ld( \R^{ 3 }, \de \mu )}
\newcommand{\Ldlim}{L^{ 2 }-\Lim}


% ####################
% Wartość bezwzględna i normy
\providecommand{\absj}[1]{\lvert #1 \rvert}
\providecommand{\absd}[1]{\left| #1 \right|}
\providecommand{\absd}[1]{\left| \, #1 \, \right|}
\newcommand{\norm}[1]{\left|\left| #1 \right|\right|}


% ####################
% Edycja tekstu
\newcommand{\tb}{\textbf}
\newcommand{\noi}{\noindent}
\newcommand{\start}{\noi \tb{--} {}}
\newcommand{\Str}[1]{\tb{Str. #1.}}
\newcommand{\StrWg}[2]{\tb{Str. #1, wiersz #2.}}
\newcommand{\StrWd}[2]{\tb{Str. #1, wiersz #2 (od dołu).}}
\newcommand{\Dow}{\tb{Dowód}}
\newcommand{\Center}[1]{\begin{center} #1 \end{center}}
\newcommand{\CenterTB}[1]{\Center{\tb{#1}}}
\newcommand{\Jest}{\tb{Jest: }}
\newcommand{\Pow}{\tb{Powinno być: }}
\newcommand{\red}[1]{{\color{red} #1}}
\newcommand{\Prze}{{\color{red} Przemyśl.}}
\newcommand{\Pop}{{\color{red} Popraw.}}
\newcommand{\Prob}{{\color{red} Problem.}}
\newcommand{\Dok}{{\color{red} Dokończ.}}
\newcommand{\Pyt}{{\color{red} Pytanie.}}
\newcommand{\Main}[1]{ \begin{center} {\LARGE \tb{#1} } \end{center} }
\newcommand{\Field}[1]{ \begin{center} {\Large \tb{#1} } \end{center} }
\newcommand{\Work}[1]{ \begin{center} {\large \tb{#1}} \end{center} }


% ##############################
% Oznaczenia dla Reeda, Simona

\newcommand{\ci}{\circ}

% Przestrzenie Hilberta
\newcommand{\lcd}{\ell_{ 2 }} % l Caligraphy Dwa
\newcommand{\SP}[2]{( #1, \, #2 )} % Scalar product.
\newcommand{\dket}[2]{\lket #1, \, #2 \rket} % Dirac ket

% % Sumy
% \newcommand{\Sum}{\sum\limits}

% % Różniczkowanie i pochodne
% \newcommand{\pr}{\partial}
% \newcommand{\de}{\mr{d}}
% \newcommand{\dd}[3]{\fr{ \de^{ #1 } { #2 } }{ \de { #3 }^{ #1 } }}
% \newcommand{\pd}[3]{\fr{ \pr^{ #1 } { #2 } }{ \pr { #3 }^{ #1 } }}

% % Całki
% \newcommand{\Int}{\int\limits}
% \newcommand{\IntA}[1]{\Int_{ -\infty }^{ +\infty } \de #1 \;}
% \newcommand{\IntB}[1]{\int_{ \R } \de #1 \;}
% \newcommand{\IntC}[1]{\Int_{ 0 }^{ +\infty } \de #1 \;}
% \newcommand{\IntCa}[2]{ \Int #1 \, \de#2 } % Juz z calkowaniem
% \newcommand{\IntFi}[2]{ \Int \de#1 \, #2 } % Bardziej Fizyczna notacja;)
% \newcommand{\IntWie}[3]{ \int_{ #1 } \de^{ #2 }#3 \; } % Wielowymiarowa.
% \newcommand{\IntDo}[5]{ \Int_{ #1 } \de^{ #2 }#3\, \de^{ #4 } #5 \; }
% \newcommand{\IntL}[3]{ \int\limits_{ { #1 } }^{ { #2 } } \d#3 \; }

% \usepackage{url}% Pozwala pisać ładnie znak ~.
% \newcommand{\vt}{$(x_1,x_{2}, \ldots, x_n)$}
% \newcommand{\vet}{(x_1,x_{2}, \ldots, x_n)}
% \newcommand{\mr}{\mathrm}
% \newcommand{\mb}{\mathbb}
% \newcommand{\mf}{\mathbf}

% \newcommand{\e}{\mr{e}}
% \newcommand{\eb}{\mf{e}}
% \newcommand{\ii}{\mr{i}}
% \newcommand{\de}{\mr{d}}
% \newcommand{\pr}{\partial}

\newcommand{\bd}[1]{\boldsymbol{#1}}
% \newcommand{\dd}[3]{\fr{ \de^{ #1 } #2 }{ \de #3^{ #1 } }}
% \newcommand{\pd}[3]{\fr{ \pr^{ #1 } { #2 } }{ \pr { #3 }^{ #1 } }}
% \newcommand{\Int}[2]{ \int #1 \, d#2 }










% ####################################################################
\begin{document}
% ####################################################################



% ########################################
\Main{Mechanika --~błędy i~uwagi} % Tytuł całego tekstu

\vspace{\spaceTwo}
% ########################################



% ####################
\Work{
  I. Newton \\
  ,,Matematyczne zasady filozofii przyrody'', \cite{Newton11} }

\CenterTB{Błędy}
\begin{center}
  \begin{tabular}{|c|c|c|c|c|}
    \hline
    & \multicolumn{2}{c|}{} & & \\
    Strona & \multicolumn{2}{c|}{Wiersz} & Jest & Powinno być \\ \cline{2-3}
    & Od góry & Od dołu &  &  \\ \hline
    16 & & 2 & Dodajęy & Dodaję \\
    % & & & & \\
    % & & & & \\
    % & & & & \\
    % & & & & \\
    \hline
  \end{tabular}
\end{center}





% ####################
\Work{
  Roman Stanisław Ingarden, Andrzej Jamiołkowski \\
  ,,Mechanika klasyczna'', \cite{IJ80} }


\CenterTB{Uwagi}

\start \Str{9--12}





% ####################
\Work{
  Lew D. Landau, Jewginij M. Lifszyc \\
  ,,Mechanika'', \cite{LL06} }


\CenterTB{Uwagi}

\start \Str{14} Podana tu grupa Galileusza składa~się tylko z~pchnięć,
co według mnie tylko zaciemnia strukturę symetrii czasoprzestrzeni
Galileusza. Pełniejsze omówienie tej grupy można znaleźć w~książce
W.~Arnolda ,,Metody matematyczne mechaniki klasycznej'' \cite{Arnold81}.

\start \Str{13} Przemyślenie jest głębokie, ale przedstawione
stanowczo zbyt krótko, aby było jasne. Spróbuję przedstawić tu pewne
jego rozwinięcie.

Przed wszystkim należy zauważyć, że należy tu rozróżnić jednorodność
i~izotropowość w sensie geometrii przestrzeni i w sensie dynamiki.
Cechy te traktowane jako cechy geometrii czasoprzestrzeni w sensie
geometrii liniowej i różniczkowej, są niezależne od układu
odniesienia. Przejdźmy teraz do problemu dynamiki. Po pierwsze z
doświadczenia wiemy, że możemy przyjąć, iż przestrzeń jest
euklidesowa, jak również że można znaleźć układ odniesienia w którym
cząstki swobodne umieszczone w przestrzeni spoczywają. \\
\start \Str{14} $\pd{}{ L }{ \mathbf{ v } }$ nie jest funkcją tylko
kwadratu prędkości. Jest to wektor o składowych
$( \pd{}{ L }{ \mathbf{ v }} )_{ i } = 2 \pd{}{ L }{ { v^{ 2 } } } v_{
  i }$, czyli zależy on jawnie od składowych prędkości. Widać jednak,
że stałość $\pd{}{ L }{ \mathbf{ v } }$ wymaga od nas stałości
$\mathbf{ v }$. Jeżeli bowiem rozpatrzymy składową $x$ wektora
(ściślej pola wektorowego) $\pd{}{ L }{ \mathbf{ v } }$, mamy warunek
na stałość tego wyrażenia dla dowolnej wartości składowej $x$:
$\pd{}{ L }{ { v^{ 2 } } } = \frac{ 1 }{ 2 \mathbf{ v }_{ x } }$.
Wyrażenie to należy zakwestionować na paru poziomach, choćby dlatego,
że jest osobliwe dla zerowych prędkości, co jest niedopuszczalne dla
fizycznej teorii. Oczywiście, jeżeli sprawdzimy również warunek na
$ y $ składową otrzymamy sprzeczny układ równań.

\start \Str{14} Należałoby podać większą dyskusję prędkości względnej
dwóch układów inercjalnych.

\start \Str{15} Jak można ściślej uzasadnić, że rzeczywiście
potrzebujemy liniowej zależności od prędkości prawej strony równania
wyrażającego równoważność między dwoma lagrażjanami? \Dok

\start \Str{22} Dyskusja ważności addytywnych zasad zachowania, ma
swoją głębię i wagę, zaciemnia ona jednak pewne szczegóły. Autorzy gdy
ją pisali musieli mieć na myśli procesy rozpraszania, nie wspomnieli
jednak, że jeśli znana jest postać oddziaływania między dwoma
cząstkami, również mamy możliwość wyciągnięcia z praw zachowania
ważnych wniosków. Np. jeśli rozpatrujemy układ dwóch cząstek i znamy
energię kinetyczną jednej z nich i energię oddziaływania, to możemy
obliczyć pewne parametry ruchu drugiej.


\CenterTB{Błędy}
\begin{center}
  \begin{tabular}{|c|c|c|c|c|}
    \hline
    & \multicolumn{2}{c|}{} & & \\
    Strona & \multicolumn{2}{c|}{Wiersz} & Jest & Powinno być \\ \cline{2-3}
    & Od góry & Od dołu &  &  \\ \hline
    56 & 12 & & poruszały się z tą samą prędkością & spoczywały \\
    % & & & & \\
    % & & & & \\
    \hline
  \end{tabular}
\end{center}
\noi
\StrWd{27}{2} \\
\Jest $S = S' + \mu \mf{V} \cdot \mf{R}' + \fr{ 1 }{ 2 } \mu V^{ 2 } t$ \\
\Pow $S = S' + \mu \mf{V} \cdot \mf{R}'( t ) - \mu \mf{V} \cdot
\mf{R}'( 0 ) + \fr{ 1 }{ 2 } \mu V^{ 2 } t$ \\





% ####################
% Tytuł i autor dzieła
\Work{
  Bogdan Skalmierski \\
  ,,Mechanika z~wytrzymałością materiałów'', \cite{Ska83} }


\CenterTB{Uwagi}

\start \Str{21} We~wzorze w~drugiej linii zamiast
\begin{displaymath}
  \sqrt{1 - \left( \tfrac{ x }{ x_{ 0 } } \right)^{ 2 } }
\end{displaymath}
powinno być
\begin{displaymath}
  \mathrm{sgn}( \cos \varphi ) \, \sqrt{1 - \left( \tfrac{ x }{ x_{ 0 } }
    \right)^{ 2 } },
\end{displaymath}
bo~wykorzystujemy jedynkę trygonometryczną by~wyrazić $\cos$ przez
$\sin$. Ponieważ w~dalszym ciągu obliczeń podnosimy ten człon do
kwadratu, ta niedokładność nie~wpływa na ostateczny wynik.

\start \Str{36} Aby wyprowadzenie wzoru (3.21) było poprawne,
potrzebujemy by
$| \dot{ \eb }_{ 1 } \cdot \eb_{ 2 } | = \dot{ \eb }_{ 1 } \cdot
\eb_{ 2 }$. Oznacza to, że~układ obraca się od~wektora $\eb_{ 1 }$
do~$\eb_{ 2 }$.


\CenterTB{Błędy}
\begin{center}
  \begin{tabular}{|c|c|c|c|c|}
    \hline
    & \multicolumn{2}{c|}{} & & \\
    Strona & \multicolumn{2}{c|}{Wiersz}& Jest & Powinno być \\ \cline{2-3}
    & Od góry & Od dołu &  &  \\ \hline
    12  & 16 & & można określić & będziemy oznaczali \\
    13  & & 10 & $\mathbf{ b }_{ y }$ & $\mathbf{ a }_{ y }$ \\
    17  &  3 & & $(\mathbf{ a } \cdot \mathbf{ c } ) \cdot \mathbf{ b }
               - ( \mathbf{ c } \cdot \mathbf{ b } ) \cdot \mathbf{ a }$
           & $(\mathbf{ a } \cdot \mathbf{ c } ) \mathbf{ b }
             - ( \mathbf{ c } \cdot \mathbf{ b } ) \mathbf{ a }$ \\
    21  &  4 & & $\frac{ y }{ { }_{ 0 } }$ & $\frac{ y }{ { y }_{ 0 } }$ \\
    21  &  4 & & $\sqrt{ 1 \:\: \left( \frac{ x }{ x_{ 0 } }
               \right)^{ 2 } } $
           & $\sqrt{ 1 - \left( \frac{ x }{ x_{ 0 } } \right)^{ 2 } } $ \\
    23  & &  9 & $+2\beta \cos( 2\omega t )$ & $-2\beta
                                             \cos( 2\omega t )$ \\
    31  & 13 & & $x_{ 2 } \pd{}{ x_{ 2 } }{ r }$
           & $\dot{ x }_{ 2 } \pd{}{ x_{ 2 } }{ r }$ \\
    31  & &  5 & $\mathbf{ a } \pd{}{ \mathbf{ r } }{ q_{ j } }$
           & $\mathbf{ a } \cdot \pd{}{ \mathbf{ r } }{ q_{ j } }$ \\
    32  &  8 & & $\pd{}{ { \mathbf{ r }^{ 2 } } }{ { q^{ j } } }$
           & $\pd{}{ \mathbf{ r } }{ { q^{ j } } }$ \\
    34  &  6 & & $\dot{ \mathbf{ e } }_{ i } \mathbf{ e }_{ j }$
           & $\dot{ \mathbf{ e } }_{ i } \cdot \mathbf{ e }_{ j }$ \\
    34  &  8 & & $\mathbf{ e }_{ i } \mathbf{ e }_{ j }$
           & $\mathbf{ e }_{ i } \cdot \mathbf{ e }_{ j }$ \\
    34  & 10 & & $\dot{ \mathbf{ e } }_{ i } \mathbf{ e }_{ j }
                + \mathbf{ e }_{ i } \dot{ \mathbf{ e } }_{ j }$
           & $\dot{ \mathbf{ e } }_{ i } \cdot \mathbf{ e }_{ j }
             + \mathbf{ e }_{ i } \cdot \dot{ \mathbf{ e } }_{ j }$ \\
    34  & 11 & & $\dot{ \mathbf{ e } }_{ i } \mathbf{ e }_{ j }$
           & $\dot{ \mathbf{ e } }_{ i } \cdot \mathbf{ e }_{ j }$ \\
    34  & 12 & & $\dot{ \mathbf{ e } }_{ i } \mathbf{ e }_{ j }$
           & $\dot{ \mathbf{ e } }_{ i } \cdot \mathbf{ e }_{ j }$ \\
    36  & 16 & & $( \xi_{ 1 } \dot{ \mathbf{ e } }_{ 1 } + \xi_{ 2 }
                \dot{ \mathbf{ e } }_{ 2 }  ) \mathbf{ e }_{ 1 }$
           & $( \xi_{ 1 } \dot{ \mathbf{ e } }_{ 1 } + \xi_{ 2 }
             \dot{ \mathbf{ e } }_{ 2 }  ) \cdot \mathbf{ e }_{ 1 }$ \\
    36  & 16 & & $( \xi_{ 1 } \dot{ \mathbf{ e } }_{ 1 } + \xi_{ 2 }
                \dot{ \mf{ e } }_{ 2 }  ) \mf{ e }_{ 2 }$
           & $( \xi_{ 1 } \dot{ \mf{ e } }_{ 1 } + \xi_{ 2 }
             \dot{ \mf{ e } }_{ 2 }  ) \cdot \mf{ e }_{ 2 }$ \\
    36  & 18 & & $\dot{ \mf{ e } }_{ 1 } \mf{ e }_{ 2 }$
           & $\dot{ \mf{ e } }_{ 1 } \cdot \mf{ e }_{ 2 }$ \\
    36  & 20 & & $\dot{ \mf{ e } }_{ 1 } \mf{ e }_{ 2 }$
           & $\dot{ \mf{ e } }_{ 1 } \cdot \mf{ e }_{ 2 }$ \\
    % & & & & \\
    \hline
  \end{tabular}
\end{center}
\noi
\StrWg{31}{13} \\
\Jest $( \dotx_{ 1 } \ib + \dotx_{ 2 } \jb )
\left( \pd{}{ x_{ 1 } }{ r } \ib + \pd{}{ x_{ 2 } }{ r } \jb \right)
\cdot \fr{ 1 }{ \bd{ 1 } }$ \\
\Pow $( \dotx_{ 1 } \ib + \dotx_{ 2 } \jb )
\cdot \left( \pd{}{ x_{ 1 } }{ r } \ib + \pd{}{ x_{ 2 } }{ r } \jb \right)
\fr{ 1 }{ | \bd{ 1 } | }$ \\





% #####################################################################
% #####################################################################
% Bibliografia
\bibliographystyle{alpha} \bibliography{Bibliography}{}


% ############################
% Koniec dokumentu
\end{document}
